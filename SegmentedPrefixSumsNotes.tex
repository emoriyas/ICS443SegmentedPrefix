\documentclass[11pt]{article}
\usepackage{latexsym}
\usepackage{amsmath}
\usepackage{amssymb}
\usepackage{amsthm}
\usepackage{epsfig}
\usepackage{graphicx}
\usepackage{amssymb}
\usepackage{algorithm}
\usepackage[noend]{algpseudocode}
\usepackage{tabu}

\newcommand{\handout}[5]{
  \noindent
  \begin{center}
  \framebox{
    \vbox{
      \hbox to 5.78in { {\bf ICS 443: Parallel Algorithms} \hfill #2 }
      \vspace{4mm}
      \hbox to 5.78in { {\Large \hfill #5  \hfill} }
      \vspace{2mm}
      \hbox to 5.78in { {\em #3 \hfill #4} }
    }
  }
  \end{center}
  \vspace*{4mm}
}

\newcommand{\lecture}[8]{\handout{#1}{#2}{#3}{Scribe: #4}{Lecture #1}}

\newtheorem{theorem}{Theorem}
\newtheorem{corollary}[theorem]{Corollary}
\newtheorem{lemma}[theorem]{Lemma}
\newtheorem{observation}[theorem]{Observation}
\newtheorem{proposition}[theorem]{Proposition}
\newtheorem{definition}[theorem]{Definition}
\newtheorem{claim}[theorem]{Claim}
\newtheorem{fact}[theorem]{Fact}
\newtheorem{assumption}[theorem]{Assumption}

% 1-inch margins, from fullpage.sty by H.Partl, Version 2, Dec. 15, 1988.
\topmargin 0pt
\advance \topmargin by -\headheight
\advance \topmargin by -\headsep
\textheight 8.9in
\oddsidemargin 0pt
\evensidemargin \oddsidemargin
\marginparwidth 0.5in
\textwidth 6.5in

\parindent 0in
\parskip 1.5ex
%\renewcommand{\baselinestretch}{1.25}

\begin{document}

\lecture{7 --- SEPTEMBER 13, 2017}{Fall 2017}{Prof.\ Nodari Sitchinava}{Eric Moriyasu, Paul Snieder}

\section{}

\section{Overview}

In the last lecture we covered iterative prefix sums and the application of of prefix sums such as string separations.

In this lecture we will introduce another parallel prefix sum method. The segmented prefix sums is a parallel sum that is partitioned into separate segments that may vary in size. Segmented prefix sum takes in a set of real numbers
and bits. The algorithm will transverse through both sets and if the corresponding bit is 0, the algorithm continues as usual. If the corresponding bit is 1 however, the algorithm will reset and start another segment in the resulting
set.

\section{Segmented Prefix Sums}

Before delving into segmented prefix sums, we must define some terminologies.

\paragraph{$\mathbb{R}$} denotes all real numbers

\paragraph{Associative operations} are operations that returns the same results if the components are reversed. Summation is an example of an associative operation, subtraction is an example of a non-associative operation.

We will use the $\oplus$ symbol to represent an associative operation. $\oplus$ takes two set of real numbers and outputs a set of real numbers. $\oplus$ is formally defined as follows: \\
let $\oplus$ be a binary associative operation.

%\cdot
  \begin{equation}
    \oplus : \mathbb{R} \oplus \mathbb{R}  \longrightarrow \mathbb{R}
  \end{equation}

With $\oplus$ we can define associativity as whenever $(a \oplus b) \oplus c = a \oplus (b \oplus c)$ is true, the operation is associative.

\paragraph{Identity $I_\oplus$} is an operand when applied to another operand in an associative function, will return the value of the other operand. Identity is formally defined as follows:

%\cdot
  \begin{equation}
    I_\oplus \oplus x = x
  \end{equation}

For example, the identity $I_\oplus$ of addition would be 0, since adding any number with 0 will result in no change. For multiplication, identity $I_\oplus$ will be 1 because multiplying any real number with 1 will result in no change.

Identity $I_\oplus$ is used in conjunction with $\oplus$ because as shown above associative operations may have different identities, so a symbol $I_\oplus$ is used to represent the identity of an operation.

\subsection{Prefix Sums}

Let us start by defining prefix sums using $\oplus$, with input X and output Y

  \begin{eqnarray*}
    Input: X &=& {x_0, x_1, \dots , x_{n-1}}\\
    Input: Y &=& {y_0, y_1, \dots , y_{n-1}}\\
    such that&:&
    \begin{cases}
      y_i = x_0, & \text{if}\ i = 0 \\
      y_{i-1} \oplus x, & \text{otherwise}
    \end{cases}
  \end{eqnarray*}

Prefix sums recursive algorithms pseudocode:

\begin{algorithm}
  \begin{algorithmic}[1]
    \State Pf(X, i)
    \If {$i = 0$} \\
      \quad\Return X[i]
    \Else \\
      \quad\Return $Pf(X, i - 1) \oplus X[i]$
    \EndIf
  \end{algorithmic}
\end{algorithm}

\subsection{Defining Segmented Prefix Sums}

Segmented Prefix sum is defined as follows

  \begin{eqnarray*}
    Input: X &=& {x_0, x_1, \dots , x_{n-1}}\\
    B &=& {b_0, b_1, \dots , b_{n-1}}\\
    Input: Y &=& {y_0, y_1, \dots , y_{n-1}}\\
    such that&:&
    \begin{cases}
      y_i = x_0, & \text{if}\ i = 0 \\
      y_{i-1} \oplus x_i, & \text{if}\ b_i = 0 \\
      x_i, & \text{if}\ b_i = 1 \\
    \end{cases}
    \\OR &:&
    \begin{cases}
      y_i = x_0, & \text{if}\ i = 0 \\
      [y_{i-1} \cdot (1 - b_i) ] \oplus x_i, & \text{otherwise}
    \end{cases}
  \end{eqnarray*}

\paragraph{We can see that these are equivalent:} 

If $b_i = 0$, in the second expression we get $[y_{i-1} \cdot (1-0)] \oplus x_i = y_{i-1} \oplus x_i$, which is what the first expression says if $b_i = I_\oplus$.\\
If $b_i = 1$, we get $[y_{i-1} \cdot (1 - 1)] \oplus x_i = I_\oplus \oplus x_i = x_i$. This is also what the first expression says if $b_i = 1$. Therefore these two expressions are equivalent.

\subsection{Operator $\otimes$} 
To help us compute segmented prefix sums, let us define a new operator, $\otimes$:

  \begin{eqnarray*}
    Let&:& \otimes: \mathbb{R} \times {0, 1} \longrightarrow \mathbb{R}\\
    x \otimes b &=& x \cdot b', where b' = boolean complement (0 \longrightarrow 1 and 1 \longrightarrow 0)\\
    &=&
    \begin{cases}
      I_\oplus, & \text{if}\ b = 1 \\
      x & \text{if}\ b = 0
    \end{cases}
  \end{eqnarray*}

We can now say that the output, Y is,\
  \begin{eqnarray*}
    y_i =
    \begin{cases}
      x, & \text{if}\ i = 0 \\
      (y_{i-1} \otimes b_i) \oplus x_i & \text{otherwise}
    \end{cases}
  \end{eqnarray*}

\paragraph{Claim 1:} $\otimes$ is not associative

$(x \otimes b_i) \otimes b_j \neq x \otimes (b_i \otimes b_j)$

\paragraph{proof:}
We can see right away that the expression on the right is not a valid expression, since $\otimes$ requires a real number as the first argument and a bit as the second argument. The expression on the right is trying to evaluate two bits, which is not valid. Therefore, $\otimes$ is not an associative operation.\\


Now using $\otimes$ in our definition for segmented prefix sums, let’s look at the first three output values:

  \begin{eqnarray*}
    y_0 &=& x_0\\
    y_1 &=& (y_0 \otimes b_1) \oplus x_1 = (x_0 \otimes b_1) \oplus x_1\\
    y_2 &=& (y_1 \otimes b_2) \oplus x_2 = [([x_0 \otimes x_1] \oplus x_1) \otimes b_2] \oplus x_2
  \end{eqnarray*}

Now we want to simplify this expression. To do this, we want to use the Distributive Property, similar to how multiplication distributes over addition. For this, we want to distribute $\otimes$ over $\oplus$.

\paragraph{Claim 2:} $\otimes$ distributes over $\oplus$

\paragraph{proof:}
Let u and v be real numbers, and b be a bit.\\
We want to prove that $(u \oplus v) \otimes b = (u \otimes b) \oplus (v \otimes b)$

  \begin{eqnarray*}
    (u \oplus v) \otimes b &=&
    \begin{cases}
      I_\oplus, & \text{if}\ b = 1 \\
      u \oplus v & \text{if}\ b = 0
    \end{cases}
    \\(u \otimes b) \oplus (v \otimes b) &=&
    \begin{cases}
      I_\oplus \oplus I_\oplus = I_\oplus, & \text{if}\ b = 1 \\
      u \oplus v & \text{if}\ b = 0
    \end{cases}
  \end{eqnarray*}

As you can see, the two expressions simplify to the same values. Therefore the claim is proven, and $\otimes$ distributes over $\oplus$.\\

Now knowing that $\otimes$ distributes over $\oplus$, we can simplify the expression for $y_2$ even further using the distributive property:

  \begin{eqnarray*}
    y_2 &=& [([x_0 \otimes b_1] \oplus x_1) \otimes b_2] \oplus x_2\\
    y_2 &=& [([x_0 \otimes b_1] \otimes b_2) \oplus (x_1 \otimes b_2)] \oplus x_2
  \end{eqnarray*}

  We want to simplify this expression further, however we already proved that $\otimes$ is not an associative operation. We can use a truth table to find an equivalent expression for $[x_0 \otimes b_2] \otimes b_2$, as seen in Table 1:

\paragraph{Table 1:} Truth table for an equivalent expression for $[x_0 \otimes b_1] \otimes b_2$

\begin{tabu} to 0.8\textwidth { | X[3] | X[3] | X[3] | X[3] | X[3] |}
 \hline
 $b_1$ & $b_2$ & $(x_0 \otimes b_1) \otimes b_2$ & $b_1 \vee b_2$ & $x_0 \otimes (b_1 \vee b_2)$ \\
 \hline
 0 & 0 & $x_0$ & 0 & $x_0$  \\
\hline
 0 & 1 & $I_\oplus$ & 1 & $I_\oplus$  \\
\hline
 1 & 0 & $I_\oplus$ & 1 & $I_\oplus$  \\
\hline
 1 & 1 & $I_\oplus$ & 1 & $I_\oplus$  \\
\hline
\end{tabu}

As seen from Table 1, columns 3 and 5 are equivalent, which means that $(x_0 \otimes b_1) \otimes b+2 = x_0 \otimes (b_1 \vee b_2)$, and we have now found an equivalent expression to $(x_0 \otimes b_1) \otimes b_2$ that we were looking for.\\

Continuing to simplify the expression for $y_2$, we now get,\\
$y_2 = ([x_0 \otimes (b_1 \vee b_2)] \oplus (x_1 \otimes b_2)] \oplus x_2$ \\
This is as far as we can simplify at the moment.\\

To conclude this section, we have now learned two things regarding the $\otimes$ operator:\\
1) $\otimes$ distributes over $\oplus$\\
2) $\otimes$ is semi-associative. “Or” ($\vee$) is a companion operator.\\

To continue simplifying, we need to define a binary associative operator.

\subsection{fat dot}

the fat dot $^{\bullet}$ is defined as follows:\\

$^{\bullet}: (\mathbb{R} \times {0,1}) \times (\mathbb{R} \times {0,1}) \longrightarrow (\mathbb{R} \times {0,1})$

Example: $(u, b_1) ^{\bullet} (v, b_2) = (w, b_3)$\\

\paragraph{Define:} $^{\bullet} : (x_i, b_i) ^{\bullet} (x_j, b_j) = ([x_i \otimes b_j] \oplus x_j , [b_1 \vee b_2])$

\paragraph{Claim 3:} For prefix sums, $y_i = x_0 \oplus x_1 \oplus \dots \oplus x_i$

For segmented prefix sums, $y_i = (x_0 , b_0) ^{\bullet} (x_1 , b_1) ^{\bullet} \cdots ^{\bullet} (x_i , b_i)$ will be true if $^{\bullet}$ is an associative operator.

In the next lecture we will prove that $^{\bullet}$ is an associative operator. Once that is proven, we can use $^{\bullet}$ in the prefix sum algorithm to compute segmented prefix sums.            

%\bibliography{mybib}
\bibliographystyle{alpha}

\end{document}
