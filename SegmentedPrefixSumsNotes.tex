\documentclass[11pt]{article}
\usepackage{latexsym}
\usepackage{amsmath}
\usepackage{amssymb}
\usepackage{amsthm}
\usepackage{epsfig}
\usepackage{graphicx}
\usepackage{amssymb}

\newcommand{\handout}[5]{
  \noindent
  \begin{center}
  \framebox{
    \vbox{
      \hbox to 5.78in { {\bf ICS 443: Parallel Algorithms} \hfill #2 }
      \vspace{4mm}
      \hbox to 5.78in { {\Large \hfill #5  \hfill} }
      \vspace{2mm}
      \hbox to 5.78in { {\em #3 \hfill #4} }
    }
  }
  \end{center}
  \vspace*{4mm}
}

\newcommand{\lecture}[8]{\handout{#1}{#2}{#3}{Scribe: #4}{Lecture #1}}

\newtheorem{theorem}{Theorem}
\newtheorem{corollary}[theorem]{Corollary}
\newtheorem{lemma}[theorem]{Lemma}
\newtheorem{observation}[theorem]{Observation}
\newtheorem{proposition}[theorem]{Proposition}
\newtheorem{definition}[theorem]{Definition}
\newtheorem{claim}[theorem]{Claim}
\newtheorem{fact}[theorem]{Fact}
\newtheorem{assumption}[theorem]{Assumption}

% 1-inch margins, from fullpage.sty by H.Partl, Version 2, Dec. 15, 1988.
\topmargin 0pt
\advance \topmargin by -\headheight
\advance \topmargin by -\headsep
\textheight 8.9in
\oddsidemargin 0pt
\evensidemargin \oddsidemargin
\marginparwidth 0.5in
\textwidth 6.5in

\parindent 0in
\parskip 1.5ex
%\renewcommand{\baselinestretch}{1.25}

\begin{document}

\lecture{NUMBER --- DATE, 2017}{Fall 2017}{Prof.\ Nodari Sitchinava}{Eric Moriyasu, Paul Snieder}

\section{Overview}

\section{Associative Operations}

Associative operations are operations that returns same results if the components are reversed. Summation is an example of an associative operation, subtraction is an example of a non-associative operation.

\subsection{Formal definition}

Associative Operation, let $\oplus$ be a binary associative operation.

%\cdot
  \begin{equation}
    \oplus : \mathbb{R} \oplus \mathbb{R}  \longrightarrow \mathbb{R}
  \end{equation}


A associative operation takes a set of real numbers as input and outputs a real number.

\section{Prefix Sum} %Down Sweep phase?

\subsection{Down sweep}

\subsection{Application of Prefix Sum}

\section{Segmented Prefix Sums}


%\bibliography{mybib}
\bibliographystyle{alpha}

\end{document}
